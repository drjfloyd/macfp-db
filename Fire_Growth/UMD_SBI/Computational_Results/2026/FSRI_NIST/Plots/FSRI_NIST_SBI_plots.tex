\documentclass[12pt]{book}

\usepackage{graphicx}
\usepackage{subcaption} % Subcaptions
\newcommand*\degC[1]{{#1}{\,\textdegree}C} % The temperature was \degC{20}

\setlength{\textwidth}{6.5in}
\setlength{\textheight}{9.0in}
\setlength{\topmargin}{0.in}
\setlength{\headheight}{0.in}
\setlength{\headsep}{0.in}
\setlength{\parindent}{0.25in}
\setlength{\oddsidemargin}{0.0in}
\setlength{\evensidemargin}{0.0in}

\pagestyle{empty}

\begin{document}

A few notes on the simulations:
\begin{itemize}
	\item All simulations included the full hood with the HRR being calculated using the same calorimetry equation as the experimental results.
	\item Other than the grid study plot, all cases run with 1\,cm resolution.
	\item Other than the angular study plot, all cases run with 400 angles.
	\item For the Spyro simulations, the ignition temperature was set at \degC{298}.
	\item For the Spyro sensitivity plots, the 25, 50, and 75 curves represent simply applying the cone burning rate for that cone exposure at ignition. 
	\item  For the Spyro sensitivity plots, the +10 and -10 indicate shifts in ignition temperature.
\end{itemize}

\begin{figure}[ht]
\centering
\begin{subfigure}[t]{0.48\textwidth}
\centering
\includegraphics[height=2.1in]{HRR_no_blowing}
\caption{\footnotesize UMD kinetics, no blowing}
\end{subfigure}
\hfill
\begin{subfigure}[t]{0.48\textwidth}
\centering
\includegraphics[height=2.1in]{HRR_blowing} 
\caption{\footnotesize UMD kinetics, blowing}
\end{subfigure}
\hfill
\begin{subfigure}[t]{0.48\textwidth}
\centering
\includegraphics[height=2.1in]{HRR_angles} 
\caption{\footnotesize UMD kinetics, blowing, change angles}
\end{subfigure}
\hfill
\begin{subfigure}[t]{0.48\textwidth}
\centering
\includegraphics[height=2.1in]{HRR_kinetics}
\caption{\footnotesize Multiple kinetics, blowing}
\end{subfigure}
\hfill
\begin{subfigure}[t]{0.48\textwidth}
\centering
\includegraphics[height=2.1in]{HRR_spyro} 
\caption{\footnotesize Spyro, blowing}
\end{subfigure}
\hfill
\begin{subfigure}[t]{0.48\textwidth}
\centering
\includegraphics[height=2.1in]{HRR_spyro_sens}
\caption{\footnotesize Spyro sensitivity using FSRI data, blowing}
\end{subfigure}
\caption{Predicted vs Measured Heat Release Rate.}
\label{SBI_UMD_HRR}
\end{figure}

\clearpage

\begin{figure}[ht]
	\centering
	\begin{subfigure}[t]{0.48\textwidth}
		\centering
		\includegraphics[height=2.1in]{Corner_HF_105}
		\caption{\footnotesize UMD kinetics, blowing}
	\end{subfigure}
	\\
	\begin{subfigure}[t]{0.48\textwidth}
		\centering
		\includegraphics[height=2.1in]{Corner_HF_105_angles} 
		\caption{\footnotesize UMD kinetics, blowing, change angles}
	\end{subfigure}
	\hfill
	\begin{subfigure}[t]{0.48\textwidth}
		\centering
		\includegraphics[height=2.1in]{Corner_HF_105_kinetics}
		\caption{\footnotesize Multiple kinetics, blowing}
	\end{subfigure}
	\hfill
	\begin{subfigure}[t]{0.48\textwidth}
		\centering
		\includegraphics[height=2.1in]{Corner_HF_105_spyro} 
		\caption{\footnotesize Spyro, blowing}
	\end{subfigure}
	\hfill
	\begin{subfigure}[t]{0.48\textwidth}
		\centering
		\includegraphics[height=2.1in]{Corner_HF_105_spyro_sens}
		\caption{\footnotesize Spyro sensitivity using FSRI data, blowing}
	\end{subfigure}
	\caption{Predicted vs Measured Corner Heat Flux, 105 s (20 s FDS avg.).}
	\label{SBI_UMD_HRR}
\end{figure}

\clearpage

\begin{figure}[ht]
	\centering
	\begin{subfigure}[t]{0.48\textwidth}
		\centering
		\includegraphics[height=2.1in]{Corner_HF_145}
		\caption{\footnotesize UMD kinetics, blowing}
	\end{subfigure}
	\\
	\begin{subfigure}[t]{0.48\textwidth}
		\centering
		\includegraphics[height=2.1in]{Corner_HF_145_angles} 
		\caption{\footnotesize UMD kinetics, blowing, change angles}
	\end{subfigure}
	\hfill
	\begin{subfigure}[t]{0.48\textwidth}
		\centering
		\includegraphics[height=2.1in]{Corner_HF_145_kinetics}
		\caption{\footnotesize Multiple kinetics, blowing}
	\end{subfigure}
	\hfill
	\begin{subfigure}[t]{0.48\textwidth}
		\centering
		\includegraphics[height=2.1in]{Corner_HF_145_spyro} 
		\caption{\footnotesize Spyro, blowing}
	\end{subfigure}
	\hfill
	\begin{subfigure}[t]{0.48\textwidth}
		\centering
		\includegraphics[height=2.1in]{Corner_HF_145_spyro_sens}
		\caption{\footnotesize Spyro sensitivity using FSRI data, blowing}
	\end{subfigure}
	\caption{Predicted vs Measured Corner Heat Flux, 145 s (20 s FDS avg.).}
	\label{SBI_UMD_HRR}
\end{figure}

\clearpage

\begin{figure}[ht]
	\centering
	\begin{subfigure}[t]{0.48\textwidth}
		\centering
		\includegraphics[height=2.1in]{Corner_HF_185}
		\caption{\footnotesize UMD kinetics, blowing}
	\end{subfigure}
	\\
	\begin{subfigure}[t]{0.48\textwidth}
		\centering
		\includegraphics[height=2.1in]{Corner_HF_185_angles} 
		\caption{\footnotesize UMD kinetics, blowing, change angles}
	\end{subfigure}
	\hfill
	\begin{subfigure}[t]{0.48\textwidth}
		\centering
		\includegraphics[height=2.1in]{Corner_HF_185_kinetics}
		\caption{\footnotesize Multiple kinetics, blowing}
	\end{subfigure}
	\hfill
	\begin{subfigure}[t]{0.48\textwidth}
		\centering
		\includegraphics[height=2.1in]{Corner_HF_185_spyro} 
		\caption{\footnotesize Spyro, blowing}
	\end{subfigure}
	\hfill
	\begin{subfigure}[t]{0.48\textwidth}
		\centering
		\includegraphics[height=2.1in]{Corner_HF_185_spyro_sens}
		\caption{\footnotesize Spyro sensitivity using FSRI data, blowing}
	\end{subfigure}
	\caption{Predicted vs Measured Corner Heat Flux, 185 s (20 s FDS avg.).}
	\label{SBI_UMD_HRR}
\end{figure}

\clearpage


\end{document}
